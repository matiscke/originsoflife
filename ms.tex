% Define document class
\documentclass[twocolumn]{aastex631}
\input{preamble}
\begin{document}

% Title
\title{Bioverse: Origins of Life}

% Author list
\author[0000-0001-8355-2107]{Martin Schlecker}
\affiliation{Steward Observatory, The University of Arizona, Tucson, AZ 85721, USA; \href{mailto:schlecker@arizona.edu}{schlecker@arizona.edu}}
\author{et al.}

\begin{abstract}
    tbd.
\end{abstract}

\section{Introduction}
\label{sec:intro}

\section{Origins of Life Scenarios}
\label{sec:ool_scenarios}

\subsection{Hydrothermal vents}
e.g., Russell+2010

\subsection{Subaerial ponds}

\subsection{Planetary redox state and evolution}
The synthesis of prebiotic compounds requires moderately to highly reduced chemical environments (Kitadai \& Maruyama 2018, Benner+2020, Sasselov+2020, Lichtenberg \& Clement 2022).

\subsection{Cyanosulfidic scenario}

\subsection{Impact trigger}
Iron-rich impactors have been suggested to intermittently provide the reduced environments favored by prebiotic chemistry (e.g., Sekine+2003, Hashimoto+2007, Kuwahara \& Sugita 2015, Genda+2017, Wogan+2023).


\section{Required conditions and predicted observables}
\label{sec:observables}

What are the environmental conditions necessary for the processes and reactions inherent to an OoL scenario to take place?
What distinct observables connected to these conditions are accessible via present and near-future remote sensing techniques?

\subsection{Hydrothermal vents}
The hydrothermal vents scenario requires a direct contact of an ocean and the planetary mantle/crust.
This requirement is not met on an ocean world with large amounts of water, where the water pressure on the ocean floor is high enough to form high-pressure ices (Noack+2016).
\todo[inline]{see discussion in Kite \& Ford 2018 Sect. 6.4}


\subsection{Subaerial ponds}
By its nature, the subaerial ponds scenario relies on rock surfaces exposed to the planetary atmosphere.
Water worlds that have their entire planetary surface covered with water contradict this requirement and do not allow for the wet-dry cycling inherent to this origin of life scenario.
The competition of tectonic stress with gravitational crustal spreading (Melosh 2011) sets the maximum possible height of mountains, which in the solar system does not exceed $\sim \SI{20}{\kilo\meter}$.
Such mountains will be permanently under water on water worlds.
Another impediment to wet-dry cycles is tidal locking of the planet as it stalls stellar tide-induced water movement and diurnal irradiation variability.

\subsection{Redox state window}
Surficial origins of life chemistries are dependent on the redox state of a planet being $\sim$neutral (not too reduced or oxidized) to allow the presence of precursor molecules like HCN. The planetary redox state leaves an imprint on its atmospheric composition and thus planet size (very reduced atmospheres are large) and spectral signatures. Connected to the cyanosulfidic scenario, the pond scenario, and the impact trigger.

\subsection{Cyanosulfidic scenario}
A strong requirement for the cyanosulfidic scenario is a sufficient Ultraviolet (UV) flux incident on the planet.
On planets that have never received significant UV fluxes, the relevant photochemistry is limited.

\subsection{Impact trigger}
Prebiotic synthesis triggered through reduced impactors that stochastically create transiently reducing or neutral atmospheres requires a certain composition of the impactors, the planet to not be in a magma ocean state (???) (Lichtenberg \& Clement 2022), and, related to this requirement, occurrence of impact events during early planetary evolution.

Observables: stochastic increase in brightness temperature, transient increase of planet size, and change of planet composition (decreasing with decreasing impact rate, ie stellar age).

\section{Bayesian Analysis}
It is instructive to consider the constraining power of a successful biosignature detection for competing OoL scenarios, which we here attempt with an analytical approach.
We may first consider the probability $\pdf(bio|H_i)$ of detecting a convincing biosignature on a planet, given a specific OoL hypothesis $H_i, \, i \in 1, 2, 3$ is true. 
This can be decomposed into the probability of abiogenesis $\pdf_\mathrm{abio}$, the occurrence rate of the planet type required by the hypothesis $\eta_i$, and the probability of detecting the biosignature on this planet type with current technology $\pdf_\mathrm{det, i}$, yielding 

\begin{align}
\label{eqn:pbio}
\pdf(bio|H_i) = \pdf_\mathrm{abio} \times \eta_i \times \pdf_\mathrm{det, i}.
\end{align}

However, what we are actually interested in is the probability of a OoL hypothesis being true, given a specific biosignature detection, $\pdf(H_i|bio)$.
To obtain this we can use Bayes' theorem, which yields 
\begin{align}
\pdf(H_i|bio) = \pdf(bio|H_i) \pdf(H_i) / \pdf(bio).
\end{align}
Putting everything together, we arrive at
\begin{align}
\label{eqn:posterior}
\pdf(H_i|bio) = \frac{\pdf(H_i)}{\pdf(bio)} \times \pdf_\mathrm{abio} \times \eta_i \times \pdf_\mathrm{det, i}.
\end{align}

Here, $\pdf(bio)$ is the prior probability of detecting the biosignature.
Following Eq.~\ref{eqn:pbio}, we can sum over all possible scenarios to obtain
\begin{align}
\pdf(bio) = \sum_{i=1}^{N} \pdf_\mathrm{abio} \times \eta_i \times \pdf_\mathrm{det, i}.
\end{align}
$\pdf(H_i)$ is the prior probability of the OoL hypothesis $H_i$.
If we assume that all OoL hypotheses are a priori equally probable, we can treat $\frac{\pdf(H_i)}{\pdf(bio)}$ in Eq.~\ref{eqn:posterior} as a normalization constant. 

In the following, we discuss the impact of the remaining variables $\pdf_\mathrm{abio}, \eta_i, $ and $\pdf_\mathrm{det, i}$ on $\pdf(H_i|bio)$.

\subsection{Probability of Abiogenesis $\pdf_\mathrm{abio}$}
Following previous work~\citep{Spiegel2012,Chen2018,Kipping2021}, we may adopt a uniform rate model for abiogenesis, i.e., assume that OoL events occur at a uniform rate.
This corresponds to a Poisson process with a rate parameter $lambda$, where we make the implicit assumption that abiogenesis occurs only via a single, unique mechanism, only once, and instantaneous.
If this event occurs within a limited time window $t$, say, between planets form around a star and when it leaves the main sequence, we have
\begin{align}
\pdf_\mathrm{abio} = 1 - \exp(-\lambda t).
\end{align}
~\\
...


\subsection{Planet occurrence rate $\eta_i$}
\subsection{Detection probability $\pdf_\mathrm{det, i}$}


\section{Exoplanet survey simulations}
\todo[inline]{Focus on Oxygen as a biomarker(?)}
\subsection{Habitable Worlds Observatory}
\subsubsection{target list}
Provisional stellar target List for the habitable Worlds Observatory~\citep{Mamajek2023}.

\subsection{Nautilus}
\subsection{LIFE} % optional

\section{Hypothesis tests}
\label{sec:hypotests}
\subsection{Possibility of hydrothermal vents}

\subsection{Availability of rock-air interfaces}

\subsection{Planetary redox state evolution}

\subsection{Planetary UV flux and its evolution}


\section{Results}
\label{sec:results}

\section{Discussion and Conclusions}
\label{sec:discussion}

''Which natural processes best explain how living matter spontaneously appears from nonliving matter?''~\citep{Malaterre2022}
%WE consider life ``as we know it''

\bibliography{bib}

\end{document}
