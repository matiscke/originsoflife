% Define document class
\documentclass[twocolumn]{aastex631}
% Some entries inspired from a preamble by Adrian Price-Whelan, https://github.com/adrn/PhaseSpiralAsteroseismology/blob/main/tex/preamble.tex

\usepackage{showyourwork}

% Latex imports
\let\tablenum\relax             % necessary for AASTeX
\usepackage{siunitx}
\sisetup{range-phrase= \text{--}, range-units=single, separate-uncertainty=true}
\sisetup{separate-uncertainty=true}
\usepackage{blindtext}          % Filler text
\usepackage{xcolor}
\usepackage[T1]{fontenc}

\usepackage[toc,nogroupskip,nohypertypes={glossary}]{glossaries} % import after hyperref
\glsdisablehyper

\newglossaryentry{HST}{name={HST}, description={Hubble Space Telescope}, first={Hubble Space Telescope (HST)}}
\newglossaryentry{NIR}{name={NIR}, description={Near-Infrared}, first={near-infrared (NIR)}}
\newglossaryentry{MIR}{name={MIR}, description={Mid-Infrared}, first={mid-infrared (MIR)}}
\newglossaryentry{FIR}{name={FIR}, description={Far-Infrared}, first={far-infrared (FIR)}}
\newglossaryentry{FUV}{name={FUV}, description={Far-Ultraviolet}, first={far-Ultraviolet (FUV)}}
\newglossaryentry{NUV}{name={NUV}, description={Near-Ultraviolet}, first={near-Ultraviolet (NUV)}}
\newglossaryentry{EUV}{name={EUV}, description={extreme-Ultraviolet}, first={extreme-Ultraviolet (EUV)}}
\newglossaryentry{SED}{name={SED}, description={Spectral energy distribution}, first={spectral energy distribution (SED)}, firstplural={spectral energy distributions (SEDs)}}
\newglossaryentry{ppm}{name={ppm}, description={parts per million}, first={parts per million (ppm)}}
\newglossaryentry{PDF}{name={PDF}, description={Probability density function}, first={probability density function (PDF)}, firstplural={probability density functions (PDF)}}
\newglossaryentry{PLATO}{name={PLATO}, description={PLAnetary Transits and Oscillations of stars}, first={PLATO~\citep[][]{Rauer2014}}}
\newglossaryentry{PSF}{name={PSF},description={Point Spread Function}, first={point spread function (PSF)}, firstplural={point spread functions (PSF)}}
\newglossaryentry{SFD}{name={SFD}, description={Size frequency distribution}, first={size frequency distribution (SFD)}, firstplural={Size frequency distributions (SFD)}}
\newglossaryentry{SNR}{name={SNR}, description={Signal-to-noise ratio}, first={signal-to-noise ratio (SNR)}}
\newglossaryentry{RV}{name={RV}, description={Radial Velocity}, first={Radial Velocity (RV)}, firstplural={Radial Velocities (RV)}}
\newglossaryentry{TTV}{name={TTV}, description={Transit Timing Variation}, first={transit timing variation (TTV)}, firstplural={transit timing variations (TTVs)}}
\newglossaryentry{TESS}{name={TESS}, description={Transiting Exoplanet Survey Satellite}, first={Transiting Exoplanet Survey Satellite~\citep[\textit{TESS},][]{Ricker2014}}}
\newglossaryentry{TIC}{name={TIC}, description={\textit{TESS} Input Catalog}, first={\textit{TESS} Input Catalog~\citep[TIC,][]{Stassun2018}}}
\newglossaryentry{JWST}{name={JWST}, description={James Webb Space Telescope}, first={James Webb Space Telescope~\citep[\textit{JWST},][]{Beichman2014}}}
\newglossaryentry{HZ}{name={HZ}, description={habitable zone}, first={habitable zone (HZ)}}
\newglossaryentry{EEC}{name={EEC}, description={exo-Earth candidate}, first={exo-Earth candidate (EEC)}, firstplural={exo-Earth candidates (EEC)}}
%\newglossaryentry{}{name={}, description={}, first={}, firstplural={}}


% package to open file containing variables
\usepackage{datatool, filecontents}
\DTLsetseparator{,}% Set the separator between the columns.

% import data
\DTLloaddb[noheader, keys={thekey,thevalue}]{variables}{variables.dat}
% Loads variables.dat with column headers 'thekey' and 'thevalue'
\newcommand{\var}[1]{\ensuremath{\DTLfetch{variables}{thekey}{#1}{thevalue}}}

% paper comments
\usepackage{comment}						 % comments that can be switched visible/invisible
\includecomment{comment}
%\specialcomment{outtake}{\begingroup\sffamily\color{gray}}{\endgroup}
%\specialcomment{note}{\begingroup\sffamily\color{red!40!green!70!blue!90}}{\endgroup}
\specialcomment{note}{\begingroup\sffamily\color{gray}}{\endgroup}
%\excludecomment{note}                       % switch notes off

%% switch TODO notes on/off
\usepackage[backgroundcolor=red!20!green!40!blue!10, textsize=tiny]{todonotes}
% \usepackage[disable]{todonotes}			% switches all todo notes to invisible
\usepackage{regexpatch}
\makeatletter
\xpatchcmd{\@todo}{\setkeys{todonotes}{#1}}{\setkeys{todonotes}{inline,#1}}{}{}
\makeatother

%% manuscript revision
%\newcommand{\rev}[1]{{\textbf{#1}}}    % on
\newcommand{\rev}[1]{{#1}}              % off

%% ... and second revision
\newcommand{\revv}[1]{{\textbf{#1}}}    % on
%\newcommand{\revv}[1]{{#1}}            % off

% ---------------------------------
% PAPER VARIABLES

% ---------------------------------
% CONSTANTS/MISSIONS/ABBREVIATIONS

% SIunitx definitions
\DeclareSIUnit\mSun{M_\odot}
\DeclareSIUnit\Msun{M_\odot}
\DeclareSIUnit\mStar{M_\star}
\DeclareSIUnit\Mstar{M_\star}
\DeclareSIUnit\mEarth{M_\oplus}
\DeclareSIUnit\Mearth{M_\oplus}
\DeclareSIUnit\rEarth{R_\oplus}
\DeclareSIUnit\Rearth{R_\oplus}
\DeclareSIUnit\year{yr}
\DeclareSIUnit\au{au}
\DeclareSIUnit\dex{dex}
\DeclareSIUnit\ppm{ppm}
\DeclareSIUnit\eV{eV}
\DeclareSIUnit\parsec{pc}
\DeclareSIUnit\photons{photons}
\DeclareSIUnit\erg{erg}

% Missions/Projects/Packages
\newcommand{\code}[1]{\texttt{#1}}
\newcommand{\project}[1]{\textsl{#1}}

\newcommand{\bioverse}{\code{Bioverse}}
\newcommand{\emcee}{\project{emcee}}

\newcommand{\rst}{\project{Nancy Grace Roman Space Telescope}}
\newcommand{\plato}{\project{PLATO}}
\newcommand{\cheops}{\project{CHEOPS}}
\newcommand{\kepler}{\project{Kepler}}
\newcommand{\ktwo}{\project{K2}}
\newcommand{\tess}{\project{TESS}}
\newcommand{\ariel}{\project{Ariel}}
\newcommand{\nautilus}{\project{Nautilus}}
\newcommand{\life}{\project{LIFE}}
\newcommand{\hwo}{\project{Habitable Worlds Observatory}}
\newcommand{\gclef}{\project{G-CLEF}}
\newcommand{\gmt}{\project{Giant Magellan Telescope}}
\newcommand{\andes}{\project{ANDES}}
\newcommand{\elt}{\project{European Extremely Large Telescope}}
\newcommand{\modhis}{\project{MODHIS}}
\newcommand{\tmt}{\project{Thirty Meter Telescope}}
\newcommand{\gaia}{\project{Gaia}}

% Stats / probability
\newcommand{\given}{\,|\,}
\newcommand{\norm}{\mathcal{N}}
\newcommand{\pdf}{\textsl{P}}
%\newcommand{\GG}{\mathbb{G}}
%\newcommand{\pbio}{\mathrm{}}

% Maths
\newcommand{\dd}{\mathrm{d}}
\newcommand{\transpose}[1]{{#1}^{\mathsf{T}}}
\newcommand{\inverse}[1]{{#1}^{-1}}
\newcommand{\argmin}{\operatornamewithlimits{argmin}}
\newcommand{\mean}[1]{\left< #1 \right>}

% Non-scalar variables
\renewcommand{\vec}[1]{\ensuremath{\bs{#1}}}
\newcommand{\mat}[1]{\ensuremath{\mathbf{#1}}}

% Unit shortcuts
\newcommand{\msun}{\ensuremath{\mathrm{M}_\odot}}
\newcommand{\mjup}{\ensuremath{\mathrm{M}_{\mathrm{J}}}}
\newcommand{\kms}{\ensuremath{\mathrm{km}~\mathrm{s}^{-1}}}
\newcommand{\mps}{\ensuremath{\mathrm{m}~\mathrm{s}^{-1}}}
\newcommand{\pc}{\ensuremath{\mathrm{pc}}}
\newcommand{\kpc}{\ensuremath{\mathrm{kpc}}}
\newcommand{\kmskpc}{\ensuremath{\mathrm{km}~\mathrm{s}^{-1}~\mathrm{kpc}^{-1}}}
\newcommand{\dayd}{\ensuremath{\mathrm{d}}}
\newcommand{\yr}{\ensuremath{\mathrm{yr}}}
\newcommand{\AU}{\ensuremath{\mathrm{AU}}}
\newcommand{\Kel}{\ensuremath{\mathrm{K}}}

% Misc.
\newcommand{\bs}[1]{\boldsymbol{#1}}

% Astronomy
\newcommand{\feh}{\ensuremath{{[{\rm Fe}/{\rm H}]}}}
\newcommand{\mh}{\ensuremath{{[{\rm M}/{\rm H}]}}}
\newcommand{\logg}{\ensuremath{\log g}}
\newcommand{\Teff}{\ensuremath{T_{\textrm{eff}}}}
\newcommand{\vsini}{\ensuremath{v\,\sin i}}
\begin{document}

% Title
\title{Bioverse: Origins of Life}

% Author list
\author[0000-0001-8355-2107]{Martin Schlecker}
\affiliation{Steward Observatory, The University of Arizona, Tucson, AZ 85721, USA; \href{mailto:schlecker@arizona.edu}{schlecker@arizona.edu}}
\author{et al.}

\begin{abstract}
    tbd.
\end{abstract}

\section{Introduction}
\label{sec:intro}

\section{Origins of Life Scenarios}
\label{sec:ool_scenarios}

\subsection{Hydrothermal vents}
e.g., Russell+2010

\subsection{Subaerial ponds}

\subsection{Planetary redox state and evolution}
The synthesis of prebiotic compounds requires moderately to highly reduced chemical environments (Kitadai \& Maruyama 2018, Benner+2020, Sasselov+2020, Lichtenberg \& Clement 2022).

\subsection{Cyanosulfidic scenario}

\subsection{Impact trigger}
Iron-rich impactors have been suggested to intermittently provide the reduced environments favored by prebiotic chemistry (e.g., Sekine+2003, Hashimoto+2007, Kuwahara \& Sugita 2015, Genda+2017, Wogan+2023).


\section{Required conditions and predicted observables}
\label{sec:observables}

What are the environmental conditions necessary for the processes and reactions inherent to an OoL scenario to take place?
What distinct observables connected to these conditions are accessible via present and near-future remote sensing techniques?

\subsection{Hydrothermal vents}
The hydrothermal vents scenario requires a direct contact of an ocean and the planetary mantle/crust.
This requirement is not met on an ocean world with large amounts of water, where the water pressure on the ocean floor is high enough to form high-pressure ices (Noack+2016).
\todo[inline]{see discussion in Kite \& Ford 2018 Sect. 6.4}


\subsection{Subaerial ponds}
By its nature, the subaerial ponds scenario relies on rock surfaces exposed to the planetary atmosphere.
Water worlds that have their entire planetary surface covered with water contradict this requirement and do not allow for the wet-dry cycling inherent to this origin of life scenario.
The competition of tectonic stress with gravitational crustal spreading (Melosh 2011) sets the maximum possible height of mountains, which in the solar system does not exceed $\sim \SI{20}{\kilo\meter}$.
Such mountains will be permanently under water on water worlds.
Another impediment to wet-dry cycles is tidal locking of the planet as it stalls stellar tide-induced water movement and diurnal irradiation variability.

\subsection{Redox state window}
Surficial origins of life chemistries are dependent on the redox state of a planet being $\sim$neutral (not too reduced or oxidized) to allow the presence of precursor molecules like HCN. The planetary redox state leaves an imprint on its atmospheric composition and thus planet size (very reduced atmospheres are large) and spectral signatures. Connected to the cyanosulfidic scenario, the pond scenario, and the impact trigger.

\subsection{Cyanosulfidic scenario}
A strong requirement for the cyanosulfidic scenario is a sufficient Ultraviolet (UV) flux incident on the planet.
On planets that have never received significant UV fluxes, the relevant photochemistry is limited.

\subsection{Impact trigger}
Prebiotic synthesis triggered through reduced impactors that stochastically create transiently reducing or neutral atmospheres requires a certain composition of the impactors, the planet to not be in a magma ocean state (???) (Lichtenberg \& Clement 2022), and, related to this requirement, occurrence of impact events during early planetary evolution.

Observables: stochastic increase in brightness temperature, transient increase of planet size, and change of planet composition (decreasing with decreasing impact rate, ie stellar age).


\section{Exoplanet survey simulations}
\todo[inline]{Focus on Oxygen as a biomarker(?)}
\subsection{Habitable Worlds Observatory}
\subsubsection{target list}
Provisional stellar target List for the habitable Worlds Observatory~\citep{Mamajek2023}.

\subsection{Nautilus}
\subsection{LIFE} % optional

\section{Hypothesis tests}
\label{sec:hypotests}
\subsection{Possibility of hydrothermal vents}

\subsection{Availability of rock-air interfaces}

\subsection{Planetary redox state evolution}

\subsection{Planetary UV flux and its evolution}


\section{Results}
\label{sec:results}

\section{Discussion and Conclusions}
\label{sec:discussion}

''Which natural processes best explain how living matter spontaneously appears from nonliving matter?''~\citep{Malaterre2022}
%WE consider life ``as we know it''

\bibliography{bib}

\end{document}
