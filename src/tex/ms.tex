%! suppress = GroupedSubSupScript
%! suppress = Ellipsis
% Define document class
\documentclass[twocolumn,twocolappendix,linenumbers]{aastex631}
%\documentclass[modern,linenumbers]{aastex631}

\input{preamble}
%! suppress = LineBreak
\begin{document}

% Title
\title{Bioverse: Origins of Life}

% Author list
\author{Martin Schlecker}
%\author[0000-0001-8355-2107]{Martin Schlecker} # issue with ORCID? https://github.com/showyourwork/showyourwork/issues/454
\affiliation{Steward Observatory, The University of Arizona, Tucson, AZ 85721, USA; \href{mailto:schlecker@arizona.edu}{schlecker@arizona.edu}}
\author{et al.}

\begin{abstract}
    tbd.
\end{abstract}

\section{Introduction}
\label{sec:intro}
\todo[inline]{Introduce OOL, the importance of planetary contexts}

\subsection{Origins of Life Scenarios and their Predictions}\label{sec:predictions}
\todo[inline]{Present widely discussed OOL scenarios and their predictions on exoplanet observables; derive testable hypotheses.}
\label{sec:ool_scenarios}
In this section, we present some of the most prominent origins of life scenarios and their observational predictions.
We focus on the necessary environmental conditions for the processes and reactions inherent to each scenario, and aim to identify distinct observables that are accessible via present and near-future remote sensing techniques.

%\subsection{Hydrothermal vents}
A widely regarded origins-of-life scenario is that abiogenesis happens in hydrothermal vents~\citep[e.g.,][]{Russel2010}.
...
The hydrothermal vents scenario requires a direct contact of an ocean and the planetary mantle/crust.
This requirement is not met on an ocean world with large amounts of water, where the water pressure on the ocean floor is high enough to form high-pressure ices (Noack+2016). % TODO Sleep et al., 2011? Sobotta et al., 2020?
\todo[inline]{see discussion in Kite \& Ford 2018 Sect. 6.4}
\todo[inline]{SR: The sealing away of the planetary interior from the ocean due to high-pressure ice layers is a common assumption for water world exoplanets (in addition to the references above, see e.g. Hu et al. 2021). I'm not convinced it is correct, because of relatively recent evidence showing the possibility of molecular assimitation into such ices and subsequent transport, e.g., https://iopscience.iop.org/article/10.1088/0004-637X/769/1/29/meta, https://iopscience.iop.org/article/10.3847/1538-4357/acb49a/meta, https://iopscience.iop.org/article/10.3847/1538-4357/aa5cfe/meta (I'm sure there are other workers in this area, this is just the group with which I am familiar). Exoplaneteers mostly uniformly accept this proposition, so it's not an unreasonable assumption if you want to run with it so long as its acknowledged and caveated reasonably; I'm just highlighting this for your attention so that you can make an informed decision. }

\textbf{Prediction:} Planets with high-pressure ices do not show biosignatures.

%\subsection{Subaerial ponds}
A different scenario is the emergence of life in hot springs or ponds that are exposed to the planet's atmosphere~\citep[e.g.,][]{Deamer2019}.
...
By its nature, the subaerial ponds scenario relies on rock surfaces exposed to the planetary atmosphere.
Water worlds that have their entire planetary surface covered with water contradict this requirement and do not allow for the wet-dry cycling inherent to this origin of life scenario.
The competition of tectonic stress with gravitational crustal spreading (Melosh 2011) sets the maximum possible height of mountains, which in the solar system does not exceed $\sim \SI{20}{\kilo\meter}$.
Such mountains will be permanently under water on water worlds.
Another impediment to wet-dry cycles is tidal locking of the planet as it stalls stellar tide-induced water movement and diurnal irradiation variability.

\textbf{Prediction:} Biosignatures occur outside the tidal locking zone and at bulk densities consistent with exposed rock.
\todo[inline]{SR: I'm not a priori sold that tidal locking means that no wet-dry cycles occur. you can still have cycling driven by transient changes in instellation due to flares, for example (e.g., https://iopscience.iop.org/article/10.3847/1538-4357/aadfd1/meta). Similarly, I wonder if 3D effects might not give rise to variability (https://iopscience.iop.org/article/10.3847/PSJ/acc9c4/meta).  I argue that it is more robust to establish a correlation between biosignatures and planets which show evidence of continents/land. I think that Ty Robinson in our department has done some work in this area, his papers might be a good starting point. Other papers which look relevant (but with which I am not familiar, as this is not my area): https://academic.oup.com/mnras/article/511/1/440/6501216, https://academic.oup.com/mnras/article/495/1/1/5733176, https://iopscience.iop.org/article/10.3847/1538-3881/aad775/meta, https://iopscience.iop.org/article/10.3847/1538-3881/aaed3a/meta, https://iopscience.iop.org/article/10.3847/1538-3881/ab2df3/meta}


%\subsection{UV flux}\label{sec:predictions:uv}
A major hypothesis in the origin of life is that UV light played a constructive role in getting life started on Earth (see Ranjan et al. 2016, 2017c; Rimmer et al. 2018; Rapf \& Vaida 2016; Pascal et al. 2012; Green et al. 2021; and sources therein).

If UV light is required to get life started, then there is a minimum planetary UV flux requirement to have an inhabited world.
This requirement is set by competitor thermal processes; if the photo-reaction does not move forward at a rate faster than the competitor thermal process(es), then the abiogenesis scenario cannot function.
On the other hand, abundant UV light vastly in excess of this threshold does not increase the probability of abiogenesis, since once the UV photochemistry is no longer limiting, some other thermal process in the reaction network will be rate-limiting process instead.
Therefore, a putative dependence of life on UV light is best encoded as a step function (see, e.g., Ranjan et al. 2017c; Rimmer et al. 2018; Rimmer, Ranjan \& Rugheimer 2021).

One origin-of-life scenario has been refined to the point where the threshold flux has been measured.
The cyanosulfidic scenario has been shown to require a mean flux of at least $ F_\mathrm{NUV, min} = \SI{6.8\pm3.6 e10}{\photons\per\centi\meter\squared\per\second\per\nano\meter}$ integrated from \SIrange{200}{280}{\nano\meter} at the surface in order to function (Rimmer et al. 2018; Rimmer et al. 2021 Astrobiology; Rimmer et al. 2023).
\todo[inline]{SR: This is an interesting number, because it is below what was available on early Earth (so this scenario could have worked on early Earth) but until recently it was below what was thought to be available on habitable zone M-dwarf exoplanets. So it was thought that identification of biosignatures on M-dwarf planets could therefore falsify the cyanosulfidic scenario, with a potential caveat for transient UV from flares. Two recent developments have complicated the picture. First, Rimmer et al. 2018 had an error in their radiative transfer routines. Correcting for this error, early M-dwarfs and highly active M-dwarfs emit enough UV to meet the Rimmer et al. 2018 criterion (Ranjan et al. 2023). Second, a recent publication argues that /all/ estimates of M-dwarf UV are underestimates, and that late M-dwarf stars have similar emission to Sunlike stars (Rekhi et al. 2023). I suspect this is incorrect, because it contradicts a lot of work from e.g. the MUSCLES collaboration and the HAZMAT project, but it's worth keeping an eye on in case it is correct after all.}
We use this threshold value as our baseline case.

\textbf{Prediction:} Past UV flux and the occurrence of biosignatures are correlated.


%\subsection{Other Processes related to the Origins of Life}
%\subsubsection{Planetary redox state and evolution}
%The synthesis of prebiotic compounds requires moderately to highly reduced chemical environments (Kitadai \& Maruyama 2018, Benner+2020, Sasselov+2020, Lichtenberg \& Clement 2022).
%...
%Surficial origins of life chemistries are dependent on the redox state of a planet being $\sim$neutral (not too reduced or oxidized) to allow the presence of precursor molecules like HCN. The planetary redox state leaves an imprint on its atmospheric composition and thus planet size (very reduced atmospheres are large) and spectral signatures. Connected to the cyanosulfidic scenario, the pond scenario, and the impact trigger.
%
%\subsubsection{Impact trigger}
%Iron-rich impactors have been suggested to intermittently provide the reduced environments favored by prebiotic chemistry (e.g., Sekine+2003, Hashimoto+2007, Kuwahara \& Sugita 2015, Genda+2017, Wogan+2023).
%...
%Prebiotic synthesis triggered through reduced impactors that stochastically create transiently reducing or neutral atmospheres requires a certain composition of the impactors, the planet to not be in a magma ocean state (???) (Lichtenberg \& Clement 2022), and, related to this requirement, occurrence of impact events during early planetary evolution.
%Suggested observables are stochastic increases in brightness temperature, transient increases of planet size, and change of planet composition (decreasing with decreasing impact rate, i.e., stellar age).


%% check https://psu.mediaspace.kaltura.com/media/David+Kipping/1_30ntfov6
\section{Bayesian Analysis}
%\todo[inline]{consider Fisher matrix (Kendall \& Stuart 1977; Tegmark 1997) analysis to determine sensitivity of a survey to a set of parameters?}
It is instructive to consider the constraining power of a successful biosignature detection for competing OoL scenarios, which we here attempt with an analytical approach.
The hydrothermal vents scenario ($H_1$) and the subaereal pond scenario ($H_2$) can be considered as mutually exclusive models, and we may study how a particular future observation of biosignatures impacts our beliefs about the relative model probabilities.

We may first consider the probability $\pdf(bio|H_i)$ of detecting a convincing biosignature on a planet, given a particular OoL hypothesis $H_i, \, i \in 1, 2$ is true.
This can be decomposed into the probability of abiogenesis in a particular environment $\pdf_\mathrm{env, i}$, the fraction of life-hosting worlds that develop atmospheric biosignatures $\pdf_\mathrm{sig}$, the probability of the planet type required by the OoL hypothesis occurring in the surveyed sample $\pdf_\mathrm{\eta, i}$, and the probability of detecting the biosignature on this planet type with current technology $\pdf_\mathrm{det, i}$, yielding

\begin{align}
    \label{eqn:pbio}
    \pdf(bio|H_i) = \pdf_\mathrm{env, i} \times \pdf_\mathrm{sig} \times \pdf_\mathrm{\eta, i} \times \pdf_\mathrm{det, i}.
\end{align}

However, what we are actually interested in is the probability of a OoL hypothesis being true, given a particular biosignature detection, $\pdf(H_i|bio)$.
To obtain this we can use Bayes' theorem, which yields
\begin{align}
    \pdf(H_i|bio) = \frac{\pdf(bio|H_i) \pdf(H_i)}{\pdf(bio)}.
\end{align}
Here, $\pdf(H_i)$ is the prior probability of the OoL hypothesis $H_i$, and $\pdf(bio)$ is the prior probability of detecting a biosignature.
% Following Eq.~\ref{eqn:pbio}, we can sum over all possible scenarios to obtain
% \begin{align}
% \pdf(bio) = \sum_{i=1}^{N} \pdf_\mathrm{env, i} \times \pdf_\mathrm{sig} \times \pdf_\mathrm{\eta, i} \times \pdf_\mathrm{det, i}.
% \end{align}

If the hypotheses $H_i$ are adjunct, i.e., their joint occurrence is impossible, one can show that
\begin{align}
    \pdf(H_i|bio) = \frac{\pdf(bio|H_i) \pdf(H_i)}{\sum_{i=1}^{N} \pdf(bio|H_i) \pdf(H_i)}.
\end{align}
Then the parameters in Eq.~\ref{eqn:pbio} that are independent of the chosen hypothesis $H_i$ eliminate and we obtain
\begin{align}
    \pdf(H_i|bio) = &\frac{\pdf_\mathrm{env, i} \pdf_\mathrm{\eta, i} \pdf_\mathrm{det, i} \pdf(H_i)}{\sum_{i=1}^{N} \pdf_\mathrm{env, i} \pdf_\mathrm{\eta, i} \pdf_\mathrm{det, i} \pdf(H_i)} \\
    \overset{\pdf(H_i) = \pdf(H_j)}{ \underset{\forall i,j \in \{1, 2\}}{=}} &\frac{\pdf_\mathrm{env, i} \pdf_\mathrm{\eta, i} \pdf_\mathrm{det, i}}{\sum_{i=1}^{N} \pdf_\mathrm{env, i} \pdf_\mathrm{\eta, i} \pdf_\mathrm{det, i}},
\end{align}
where in the last step we made the implicit assumption that all OoL hypotheses are a priori equally probable.

If we take the ratio of these posteriors for our two independent hypotheses $H_1$ and $H_2$, we get the \textit{Bayes Factor}
\begin{align}
    \label{eq:bayesfactor}
    \frac{\pdf(H_1|bio)}{\pdf(H_2|bio)} = \frac{\pdf_\mathrm{env, 1} \pdf_\mathrm{\eta, 1} \pdf_\mathrm{det, 1}}{\pdf_\mathrm{env, 2} \pdf_\mathrm{\eta, 2} \pdf_\mathrm{det, 2}},
\end{align}
which quantifies the evidence of the data arising from $H_1$ versus $H_2$.

% Putting everything together, we arrive at
% \begin{align}
% \label{eqn:posterior}
% \pdf(H_i|bio) = \frac{\pdf(H_i)}{\pdf(bio)} \times \pdf_\mathrm{env, i} \times \pdf_\mathrm{sig} \times \pdf_\mathrm{\eta, i} \times \pdf_\mathrm{det, i}.
% \end{align}
% If we assume that all OoL hypotheses are a priori equally probable, we can treat $\frac{\pdf(H_i)}{\pdf(bio)}$ in Eq.~\ref{eqn:posterior} as a normalization constant.
In the following, we discuss the impact of the remaining variables $\pdf_\mathrm{env, i}$, $\pdf_\mathrm{\eta, i}$ and $\pdf_\mathrm{det, i}$ on the Bayes factor.

\subsection{Required environment $\pdf_\mathrm{env, i}$}
% Following previous work~\citep{Spiegel2012,Chen2018,Kipping2021}, we may adopt a uniform rate model for abiogenesis, i.e., assume that OoL events occur at a uniform rate.
% This corresponds to a Poisson process with a rate parameter $lambda$, where we make the implicit assumption that abiogenesis occurs only via a single, unique mechanism, only once, and instantaneous.
% If this event occurs within a limited time window $t$, say, between planets form around a star and when it leaves the main sequence, we have
% \begin{align}
% \pdf_\mathrm{env, i} = 1 - \exp(-\lambda t).
% \end{align}
% ~\\
\todo[inline]{The planet should be in the liquid-water HZ}
...
Let us assume that $H_1$ only requires a minimum bulk density $\rho_1$, such that $\pdf_\mathrm{env, 1} \rightarrow 1$ for $\rho >> \rho_1$ and $\pdf_\mathrm{env, 1} \rightarrow 0$ for $\rho << \rho_1$.
On the other hand, $H_2$ requires exposed land and thus a small water mass fraction.
We may implement this in the same way as above but with a minimum bulk density $\rho_2 > \rho_1$.
Furthermore, there is a requirement that the tidal locking timescale may not be so small that the planet is likely tidally locked at observation.
This translates to imposing a minimum semimajor axis $a_2$.

To approximate these thresholds including their expected intrinsic fuzziness, we model them with logistic sigmoid functions
\begin{align}
    \pdf_\mathrm{env, 1} &= \frac{1}{1+\exp[-(C  (\rho - \rho_1))]} \quad \mathrm{and}\\
    \pdf_\mathrm{env, 2} &= \frac{1}{1+\exp[-(C  (\rho - \rho_2))]} \times \frac{1}{1+\exp[-(C  (a - a_2))]},
\end{align}
where $C$ is a compression factor characterizing the steepness of the sigmoid function.
\begin{figure}[ht!]
    \script{bayes_rho-a.py}
    \begin{centering}
        \includegraphics[width=\linewidth]{figures/analytic/Penv.pdf}
        \caption{
            Probability of abiogenesis for the OoL hypotheses $H_1$ and $H_2$ as a function of planet bulk density and semi-major axis.
            $H_1$ only requires large enough densities to exclude deep global water oceans.
            $H_2$ requires a higher minimum density due to the exposed land requirement, and small semi-major axes are excluded to prevent tidal locking.
        }
        \label{fig:Penv}
    \end{centering}
\end{figure}
Figure~\ref{fig:Penv} shows the corresponding $\pdf_\mathrm{env}$ factors and where their regions of high probability overlap.

\subsection{Planet occurrence rate $\pdf_{\eta}$}
We model $\pdf_{\eta, i} (a, \rho)$ following the suggested broken power-law occurrence rates from NASA’s Exoplanet Program Analysis Group chartered Science Analysis Group 13 (SAG 13)~\citep[see][]{Bixel2021} and converting between semi-major axis and period, and between bulk density and radius assuming Earth-like orbits and compositions. \todo[inline]{this is an oversimplification.}
\begin{figure}[ht!]
    \script{bayes_rho-a.py}
    \begin{centering}
        \includegraphics[width=\linewidth]{figures/analytic/Peta.pdf}
        \caption{
            Planet occurrence rate density assuming SAG~13 occurrence rates.
        }
        \label{fig:Peta}
    \end{centering}
\end{figure}
The resulting occurrence rate density has a strong preference for low-density planets on short orbits (see Fig.~\ref{fig:Peta}).
\todo[inline]{This assumes the same occurrence rate for both hypotheses. Is this sensible? If yes, it does not impact the Bayes factor (Eq. \ref{eq:bayesfactor}); we could bring it only later to see where high Bayes factor and planet occurrence overlap.}

\subsection{Information content of a biosignature detection}
We may now evaluate Eqn.~\ref{eq:bayesfactor} to measure the information content with respect to favoring $H_1$ versus $H_2$ depending on the position of a planet with a confirmed biosignature detection in density-orbital distance space.
\begin{figure}[ht!]
    \script{bayes_rho-a.py}
    \begin{centering}
        \includegraphics[width=\linewidth]{figures/analytic/bayes_rho-a.pdf}
        \caption{
            Bayes factor (Eqn.~\ref{eq:bayesfactor}) evaluated at different bulk densities and orbital distances.
            Contour levels reflect the empirical scale for strength of evidence suggested by Jeffreys et al. 19XX.
            There, $\ln(\mathcal{B}) = 2.5$ corresponds to ``moderate'' evidence, and $\ln(\mathcal{B}) = 5 $ corresponds to ``strong'' evidence.
            Only short orbits ($a \lessapprox \SI{0.03}{\au}$) and low bulk densities ($\rho \lessapprox \SI{0.6}{\rho_\oplus}$) allow a selection between the proposed models; they lead to a strong preference for hypothesis $H_1$.
            There is no region in this parameter space that provides strong evidence for $H_2$.
        }
        \label{fig:bayes_rho-a}
    \end{centering}
\end{figure}
Figure~\ref{fig:bayes_rho-a} shows the logarithm of the Bayes factor in this space, providing a scale for evaluating the strength of evidence to prefer one of the proposed models.
Only a small region allows for a significant model selection: While short orbits ($a \lessapprox \SI{0.03}{\au}$) and low bulk densities ($\rho \lessapprox \SI{0.6}{\rho_\oplus}$) strongly support $H_1$, no combination of bulk density and orbital distance provide strong evidence for $H_2$ over $H_1$ without additional information.
 \todo[inline]{factor in Detection probability P\_det}
 \todo[inline]{TODO: test sensitivity of this result on the assumed function and thresholds for P\_env}




% \subsection{biosignature fraction $\pdf_\mathrm{env, i}$}
% Limiting ourselves to the search for \textit{life-as-we-know-it}, we may assume that there are atmospheric biomarkers present on a planet after the abiogenesis event and until global extinction.

%\subsection{Fractional planet occurrence rate $\pdf_\mathrm{\eta, i}$}
%As the different OoL hypotheses do not all work on the same planet types, we may study the impact of the fractional occurrence rates of different planet types on the posterior probability $\pdf(H_i|bio)$.
%For simplicity, let's consider only planets that allow at least one of the scenarios.
%We also ignore any influence of other planets in the same system, e.g. an outer gas giant that itself does not develop life~\citep{Schlecker2021a} or panspermia scenarios CITE.
%We may then distinguish between:
%\begin{enumerate}
%    \item Earths, i.e., limited-water terrestrial planets in the liquid water habitable zone ($H1, H2$). These planets with roughly Earth-like water mass fractions support both the existence of submarine hydrothermal vents ($H1$) and hydrothermal fields with wet/dry cycles ($H2$). Limits in exoplanet observables to this planet type are their orbital distance (both scenarios require liquid water; at least for FGK stars this requirement also puts the planet outside of the tidal locking zone ($H2$)), and bulk density ($H1$ and $H2$ require limited water fractions).
%    \item ``shallow ocean'' water worlds ($H1$). These planets have no land surface exposed to the atmosphere, thus excluding the subaerial pond scenario.
%    \item ``deep ocean'' water worlds. Through the development of high-pressure ices, these planets do not support any of the considered OoL scenarios.
%\end{enumerate}
%
%
%


\subsection{Detection probability $\pdf_\mathrm{det, i}$}



\subsection{Discussion of Bayesian analysis}

\begin{note}
    shallow ocean planets are vulnerable to water loss through high-energy radiation, limiting the time window for habitability especially if no geochemical feedbacks exist~\citep{Kite2018}.
\end{note}
\todo[inline]{discuss host star spectral type dependencies, e.g., abiogenesis time window for G stars ($\lessapprox 10 Gyr$, MS lifetime) or M dwarfs. \citep[e.g.,][]{Spiegel2012}}








%\subsection{Origins of Life Hypotheses}
...
\begin{figure*}
    \begin{centering}
        \includegraphics[width=\hsize]{figures/hypotheses.pdf}
        \caption{Population-level hypothesis and null hypothesis on UV irradiance derived from the cyanosulfidic scenario.}
        \label{fig:hypotheses}
    \end{centering}
\end{figure*}
Figure~\ref{fig:hypotheses} shows the hypothesis and null hypothesis derived from the predictions of the cyanosulfidic scenario.
...


\section{Methods}
\label{sec:hypotests}
%\todo[inline]{Briefly introduce Bayesian model comparison, then present the particular hypotheses in their mathematical form.}


\subsection{Fraction of inhabited planets with detectable biosignatures}
Presumably, not all habitable worlds are inhabited and not all inhabited worlds develop detectable biosignatures.
The fraction of \glspl{EEC} that are both inhabited and harbor detectable biosignatures at the time when we observe them remains speculative; we aggregate them in the unitless parameter $f_\mathrm{life}$.

%\subsection{H1: Life only originates in hydrothermal vents}
%The hydrothermal vent scenario does not allow oceans deep enough to form an impenetrable layer of high-pressure ice on its floor.
%The resulting allowed parameter space is described by a lower limit on the bulk density. \todo[inline]{+ exclude atmospheric signature for water worlds?}
%...
%
%\subsection{H2: Life only originates in subaerial ponds with wet/dry cycles}
%We parametrize the required exposed land surface in this scenario as a lower limit in bulk density that is higher than for H1.
%Further, the tidal locking timescale of the planet may not be smaller than the age of the system.
%...
%
%\subsection{H3: Life only originates on planets with particular UV irradiance}
\todo[inline]{\citet{Guenther2020} relate U-band energy to bolometric flux.}
\todo[inline]{OPTIONAL: ``We further test the scenario of a linear correlation of past UV flux and biosignature occurrence rate.
This test requires the detection of multiple biosignatures.''
~\\
Test for negative correlations as well?}
Here, we conduct a theoretical experiment on the UV irradiance requirement (Sect.~\ref{sec:predictions}) by relating the occurrence of life on an exo-earth candidate with a minimum past quiescent stellar UV flux, focusing on the prebiotically interesting \gls{NUV} range from \SIrange{200}{280}{\nano\meter}. \todo[inline]{SUKRIT: got a good reference for this?}  %~\citep{RanjanSasselov2016}.?
Our concrete hypothesis shall be that life only occurs on planets that at some point in their history have received such radiation exceeded a minimum flux $F_\mathrm{NUV, min}$.



\subsection{Semi-analytical approach}\label{sec:met-semianalytical}
First, we apply a Bayes Factor Design Analysis~\citep{Schoenbrodt2018} to assess the expected probabilities of obtaining true negative or true positive evidence for the hypothesis above, as well as the probability for misleading or inconclusive evidence, under idealized conditions.
This serves as a first-order estimate of the information content of a survey, before we take into account impacts from exoplanet demographics, sample selection, and survey strategy.

Let our observable be the inferred past \gls{NUV} flux of the planet $F_\mathrm{NUV}$.
Under Hypothesis $H_1$ (Equation~\ref{eq:hypothesis-uv}), there exists a special unknown value of $F_\mathrm{NUV}$, noted $F_\mathrm{NUV, min}$ such that

\begin{align}
    P(L|\theta,H_1) &=  f_\mathrm{life} \quad \text{if } \theta>F_\mathrm{NUV, min}\\
    P(L|\theta,H_1) &=  0               \quad  \text{otherwise}
\end{align}

where $f_\mathrm{life}$ is the unknown probability of abiogenesis.
The corresponding null hypothesis is that there exists no such special value of $F_\mathrm{NUV}$ and that
\begin{equation}
P(L|\theta,H_\mathrm{null}) = f_\mathrm{life}.
\end{equation}

If we now define a sample of size $n$ as $X=\{F_\mathrm{NUV, i},L_i\}_{i \in [1,n]}$ where $L_i$ is equal to 1 if life is detected and 0 otherwise, we can calculate the evidence for hypothesis $H_i$ against $H_j$ through the Bayes factor
\begin{equation}
BF_{H_i,H_j} = \frac{P(X|H_i)}{P(X|H_j)},
\end{equation}
with $P(X|H_i)$ and $P(X|H_j)$ likelihoods of obtaining the sample $X$ under either hypothesis.

If we define $k=\sum L_i$ and denote $Y$ the random variable that describes it, $H_\mathrm{null}$ represents the likelihood that the number of planets with life in the sample follows the binomial distribution
\begin{equation}
P(Y=k|H_\mathrm{null}) = \binom{n}{k}f_\mathrm{life}^k(1-f_\mathrm{life})^{n-k}.
\end{equation}

Under $H_1$, $Y$ also follows a binomial distribution, however it is conditioned by $n_{\lambda}=Card(\{F_\mathrm{NUV, i} \text{ if } F_\mathrm{NUV, i}>F_\mathrm{NUV, min}\}_{i \in [1,n]})$ the number of values of $F_\mathrm{NUV}$ in the experiment that exceed $F_\mathrm{NUV, min}$
\begin{equation}
\label{eq:semian:likelihoodH1}
P(Y=k|H_1) = \binom{n_{\lambda}}{k}f_\mathrm{life}^k(1-f_\mathrm{life})^{n_{\lambda}-k}.
\end{equation}

Hence,
\begin{equation}\label{eq:bayes_factor}
BF_{H_1,H_\mathrm{null}} = \frac{P(Y=k|H_1)}{P(Y=k|H_\mathrm{null})} = \frac{\binom{n_\lambda}{k}}{\binom{n}{k}}(1-f_\mathrm{life})^{n_{\lambda}-n}.
\end{equation}


Given a sample of planets, where for some of them we have convincing biosignature detections but remaining agnostic on $f_\mathrm{life}$: What evidence for $H_\mathrm{1}$ and $H_\mathrm{null}$ can we expect to get?
Our Bayes factor (Equation~\ref{eq:bayes_factor}) is determined by the unknown variables $f_\mathrm{life}$ and $F_\mathrm{NUV, min}$, as well as the number of planets with biosignature detections in the sample $k$.
To compute the distribution of evidences, we repeatedly generated samples under $H_\mathrm{1}$ and $H_\mathrm{null}$ and computed the Bayes factors $BF_{H_1,H_\mathrm{null}}$ and $BF_{H_\mathrm{null}, H_1}$.
We then evaluated the fraction of Monte Carlo runs in which certain evidence thresholds~\citep{Jeffreys1939} were exceeded.


%\subsection{Bayesian hypothesis testing with \bioverse}


\subsection{Exoplanet survey simulations with Bioverse}
A real exoplanet survey will be subject to observational biases, sample selection effects, and the underlying demographics of the planet sample.
To assess the information gain of a realistic exoplanet survey, we employed Bioverse~\citep{Bixel2021}, a framework that integrates multiple components including statistically realistic simulations of exoplanet populations, a survey simulation module, and a Bayesian hypothesis testing module to evaluate the statistical power of different observational strategies.
The general approach is as follows:
\begin{enumerate}
\item \textbf{Exoplanet population synthesis:} We populate the Gaia Catalogue of Nearby Stars~\citep{Smart2021} with synthetic exoplanets whose orbital parameters and planetary properties reflect our current understanding of exoplanet demographics~\citep{Bergsten2022}.
    \item \textbf{Survey simulation:} We simulate the detection and characterization of these exoplanets with a hypothetical survey, taking into account the survey's sensitivity, target selection, and observational biases.
To model the sensitivity of the information gain of a proposed mission to sample selection and survey strategy, we conduct survey simulations with Bioverse using different sample sizes and survey strategies.
    \item \textbf{Hypothesis testing:} We employ Bayesian statistical methods to evaluate the likelihood that a given survey would detect a specified statistical trend in the exoplanet population and estimate the precision with which the survey could constrain the parameters of that trend.
    A common definition of the null hypothesis $H_0$, which is also applied here, is that there is no relationship between the independent variable (here: maximum NUV flux) and the dependent variable (here: biosignature occurrence).
    The alternative hypothesis $H_1$ proposes a specific relationship parameterized by $\theta$.
    If the dependent variable is of binary nature such as in the case of biosignature detections, we model the likelihood function as
    \begin{equation}
\mathcal{L}(y \mid \boldsymbol{\theta})=\prod_i^N\left[y_i h\left(\boldsymbol{\theta}, x_i\right)+\left(1-y_i\right)\left(1-h\left(\boldsymbol{\theta}, x_i\right)\right)\right],
\end{equation}
where here $y_i \in \{0,1\}$ is the biosignature detection variable, $x_i$ is the maximum NUV flux, and $h(\boldsymbol{\theta}, x_i)$ is the probability of detecting a biosignature given the maximum NUV flux and the model parameters $\boldsymbol{\theta}$.
We then use a nested sampling method~\citep{Speagle2020} to compute the Bayesian evidence for the null and alternative hypotheses and estimate the strength of evidence for the alternative hypothesis.

To determine the diagnostic capability of a given survey, Bioverse runs multiple iterations of the simulated survey and calculates the fraction of realizations that successfully reject the null hypothesis.
We use this metric, known as the statistical power, to quantify the potential information content of the survey, identify critical design trades, and find strategies that maximize the survey's scientific return.
The posterior samples obtained from the nested sampling runs further allows us to estimate the precision with which the survey could constrain the parameters of the hypothesized trend.
\end{enumerate}


\subsubsection{Simulated star and planet sample}
We generated two sets of synthetic exoplanet populations, one for FGK-type stars and one for M-type stars.
The stellar samples are drawn from the Gaia Catalogue of Nearby Stars~\citep{Smart2021} with a maximum Gaia magnitude of \var{M_G_max} and a maximum stellar mass of \var{M_st_max} \SI{}{\Msun}.
We included stars out to a maximum distance $d_{\max}$ that depends on the required planet sample size.
Planets were generated and assigned to the synthetic stars following the occurrence rates and size/orbit distributions of \citet{Bergsten2022}.
Following \citet{Bixel2021}, we considered only transiting \glspl{EEC} with radii $0.8\, S^{0.25} < R < 1.4 $ that are within the habitable zone (see Section~\ref{sec:met-hz}).
The lower limit was suggested as a minimum planet size to retain an atmosphere~\citep{Zahnle2017}.
To generate planet samples larger than what the stellar catalog in combination with these occurrence rates yields, we scaled up the occurrence rates by a constant factor that yields the desired number of planets.
This was in particular necessary for the FGK sample, where the rate of transiting planets that occupy the habitable zone is low.

For all survey simulations and hypothesis tests, we repeated the above in a Monte Carlo fashion to generate randomized ensembles of synthetic star and planet populations~\citep[compare][]{Bixel2021}.



\subsubsection{Habitable zone occupancy and UV flux}\label{sec:met-hz}
To test the UV hypothesis, we require that life occurs only on planets with sufficient past UV irradiation exceeding $F_\mathrm{NUV, min}$.
Further, we require this flux to have lasted for a minimum duration $\Delta T_\mathrm{min}$ to allow for a sufficient ``origins timescale''~\citep{Rimmer2023}.
All common Origin of Life scenarios require water as a solvent;
we thus consider only rocky planets that may sustain liquid water on their surface, i.e., that occupy their momentary \gls{HZ} during the above period.

Different formulations of habitable zones as regions around a star where a planet with Earth's atmospheric composition can maintain liquid water on its surface exist~\citep[e.g.,][]{MolLous2022,Spinelli2023,Tuchow2023}. CITE! Ramirez \& Kaltenegger 2017, 2018
Here, we adopt the popular estimates of \citet{Kasting1993} and \citet{Kopparapu2013,Kopparapu2014} that define a temperate zone between the runaway greenhouse transition CITE! and the maximum greenhouse limit CITE!.
We use the parametrization in \citet{Kopparapu2014} to derive luminosity and planetary mass-dependent edges of the \gls{HZ} $a_\mathrm{inner}$ and $a_\mathrm{outer}$.

% detecting methanogenic life: Sandora2020, Wang2018, Seeburger2023
A commonly discussed biosignature is molecular Oxygen (O$_2$), which on Earth emerged as a byproduct of photosynthesis during the Proterozoic era.
No individual component of an atmosphere has been identified as a reliable biosignature in isolation~\citep{Seager2016}, and the detection of Oxygen alone will certainly not be sufficient to confirm the presence of life.
Nevertheless, we focus here on detecting this key absorber as it represents the general inherent observational challenges and trends.

To determine \gls{HZ} occupancy, we interpolated the stellar luminosity evolution grid of \citet{Baraffe1998} using a Clough Tocher interpolant~\citep[][see left panel of Figure~\ref{fig:hz_nuv_evo}]{Nielson1983,Alfeld1984} to compute the evolution of the inner (runaway greenhouse) and outer (maximum greenhouse) edges as a function of planet mass and stellar spectral type~\citep{Kopparapu2014}.
This provides each planet's epochs within and outside the \gls{HZ}.
For the \gls{NUV} flux, we use the age- and stellar mass-dependent \gls{NUV} fluxes in the \gls{HZ} obtained by \citet{Richey-Yowell2023}.
We linearly interpolate in their measured grid, where we convert spectral type to stellar mass using the midpoints of their mass ranges (\SI{0.75}{\Msun} for K~stars, \SI{0.475}{\Msun} for early-type M~stars,  and \SI{0.215}{\Msun} for late-type M~stars).
Outside the age and stellar mass range covered in \citet{Richey-Yowell2023}, we extrapolate using nearest simplex (see right panel of Figure~\ref{fig:hz_nuv_evo}).

We then determined which planets were both in the \gls{HZ} and had \gls{NUV} fluxes above $F_\mathrm{NUV, min}$ for $\Delta T_\mathrm{min} \geq \var{deltaT_min}\,\SI{}{\mega\year}$.
We assigned the development of life to a random fraction $f_\mathrm{life}$ of all temperate planets fulfilling these requirements.


For the hypothesis tests, we then define our alternative hypothesis as
\begin{equation}\label{eq:hypothesis-uv}
    H_{1} = f_\mathrm{life} (\theta, F_\mathrm{NUV}) =
        \begin{cases}
            0 , & F_\mathrm{NUV} < F_\mathrm{NUV, min}\\
            f_\mathrm{life}, & F_\mathrm{NUV} \geq F_\mathrm{NUV, min} \,\mathrm{and\, in\, HZ} \, \mathrm{for} \, \Delta t \geq \var{deltaT_min} \,\SI{}{\mega\year}
        \end{cases}
\end{equation}
and the corresponding null hypothesis
%\begin{equation}
$H_{null} = f_\mathrm{life} (\theta),$
%\end{equation}
i.e., no correlation with UV flux.
We imposed log-uniform priors on the parameters $\boldsymbol{\theta}$, sampling $f_\mathrm{life}$ from a log-uniform distribution between $10^{-3}$ and $1$ and $F_\mathrm{NUV, min}$ from a log-uniform distribution between $10^{1}$ and $10^{5}$.


\begin{figure}
    \script{inhabited_FGKM.py}
    \begin{centering}
        \includegraphics[width=\hsize]{figures/inhabited_FGKM}
        \caption{Host stars of all transiting \glspl{EEC} and inhabited planets in a simulated transit survey.
        In the FGK sample, most \glspl{EEC} and all inhabited planets orbit K~dwarfs.
        In an M~dwarf sample of the same size, the fraction of inhabited planets is larger.
        }
        \label{fig:inhabited_FGKM}
    \end{centering}
\end{figure}

As shown in Figure~\ref{fig:inhabited_FGKM}, the majority of \glspl{EEC} orbit lower-mass stars.
The fraction of inhabited planets is highest in the M~dwarf sample due to the higher \gls{NUV} fluxes in the \gls{HZ} of these stars.


\begin{figure*}
    \script{hz_nuv_evo.py}
    \begin{centering}
        \includegraphics[width=\hsize]{figures/hz_nuv_evo}
        \caption{Interpolated stellar luminosity evolution (left) and evolution of the \gls{NUV} flux in the \gls{HZ} (right) as a function of host star mass.
        The scatter plots show age and host star mass of the transiting planets in the synthetic FGK sample; crosses denote the estimated \gls{NUV} values in \citet{Richey-Yowell2019}. % with \gls{EEC} highlighted in green.
        A few example tracks for an example threshold flux of $F_\mathrm{NUV, min} = \var{NUV_thresh}\,\SI{}{\erg\per\second\per\centi\meter\squared}$ are shown; extended overlap of \gls{HZ} occupancy and high \gls{NUV} flux (green sections) fulfills our requirement for abiogenesis.
        Planet~1 is an \gls{EEC} that never receives sufficient \gls{NUV} flux for abiogenesis.
        Planet~2 and Planet~3 enter the \gls{HZ} at different times and receive sufficient \gls{NUV} flux for different durations until their respective host star evolves below the threshold.
        }
        \label{fig:hz_nuv_evo}
    \end{centering}
\end{figure*}





\subsubsection{Transit survey simulations}
With the synthetic star and planet samples generated, we simulated observations of these planets with a hypothetical transit survey.
Using Bioverse's survey module, we simulated noisy measurements of key observables assuming a capable transit survey that can characterize a large sample with high photometric precision.
We roughly followed the mission parameters of the Nautilus mission concept~\citep{Apai2019,Apai2022} and measured planetary instellation (for \gls{HZ} occupancy) with a precision of \var{nautilus_S} and host star effective temperature with a precision of \var{nautilus_T_eff_st}~K.
We assumed that the maximum past \gls{NUV} flux a planet received can be determined within a precision of \var{nautilus_max_nuv}.
To marginalize over choices of biosignatures and their detectability, which are beyond the scope of this study, we assumed that any inhabited planet would show a biosignature detectable by the survey.

%\subsubsection{Direct imaging survey (e.g., Habitable Worlds Observatory, LIFE)}
%\subsubsubsection{target list}
%\todo[inline]{Provisional stellar target List for the habitable Worlds Observatory:~\citep{Mamajek2023}} % read like in https://github.com/mkenworthy/HWObows/tree/40b1e5f2bc2bbbe8c999b097c80ca5a8cda0aeae/src/scripts



\section{Results}
\label{sec:results}
%\subsection{Information content in mass-density space}
%\subsection{Information content in tidal locking timescale-density space}

%\subsection{Correlation of past UV flux and biosignature occurrence}\label{sec:results_uv}

\subsection{Semi-analytical assessment}\label{sec:results-semianalytical}
In Section~\ref{sec:met-semianalytical} we computed the probability for true positive evidence for $H_\mathrm{1}$ and $H_\mathrm{null}$ respectively.
Figure~\ref{fig:semian_true_evidence} shows how these evidences are distributed for sample sizes \var{semian_Nsamp1} and \var{semian_Nsamp2}, and how likely we are to obtain strong evidence ($BF_{H_i, H_j}$ > 10).
For $n = \var{semian_Nsamp1}$, strong true evidence for $H_\mathrm{1}$ ($H_\mathrm{null}$) can be expected in $\sim \SI{30}{\percent}$ ($\sim \SI{40}{\percent}$) of all random experiments.
In the majority of cases, the outcome of the survey will be inconclusive.
\begin{figure*}
    \script{semian_true_evidence.py}
    \begin{centering}
        \includegraphics[width=\hsize]{figures/semian_true_evidence}
        \caption{Probability to obtain true strong evidence. Left: evidence levels for $H_\mathrm{1}$ and $H_\mathrm{null}$ under sample sizes $n = \var{semian_Nsamp1}$ (solid) and $n = \var{semian_Nsamp2}$ (dashed). The vertical lines denote the thresholds for ``strong'' evidence, $BF_{H_i, H_j}$ > 10, and ``extreme'' evidence, $BF_{H_i, H_j}$ > 100. Right: Probability of true strong evidence for $H_\mathrm{1}$ as a function of sample size $n$.}
        \label{fig:semian_true_evidence}
    \end{centering}
\end{figure*}
The situation improves with larger samples: for $n = \var{semian_Nsamp2}$, \SI{80}{\percent} of random samples permit conclusive inference (strong true evidence) under either $H_\mathrm{1}$ or $H_\mathrm{null}$.

The expected resulting evidence further depends on the a priori unknown abiogenesis rate $f_\mathrm{life}$ and on the \gls{NUV} flux threshold.
\begin{figure*}
    \script{semian_true_evidence.py}
    \begin{centering}
        \includegraphics[width=\hsize]{figures/semian_evidence-grid}
        \caption{Probability of obtaining true strong evidence for different abiogenesis rates, \gls{NUV} flux thresholds, and sample sizes. For each of these parameters, higher values increase the probability of yielding strong evidence.}
        \label{fig:semian_evidence-grid}
    \end{centering}
\end{figure*}
Figure~\ref{fig:semian_evidence-grid} illustrates this dependency: For very low values of either parameter, samples drawn under the null or alternative hypotheses are indistinguishable and the Bayesian evidence is always low.
Both higher $f_\mathrm{life}$ and higher \gls{NUV} flux thresholds increase the probability of obtaining strong evidence.
Larger sample sizes enable this at lower values of these parameters.


So far, we have assumed random, uniform distributions of $f_\mathrm{life}$, $F_\mathrm{NUV, min}$, and $F_\mathrm{NUV}$. % Actually it is more than just NUV, it is whatever index that integrates NUV and HZ requiremetns into a single scalar variable for which a threshold exist
A high biosignature detection rate $f_\mathrm{life}$ increases the evidence (cmp. Equation~\ref{eq:bayes_factor}) but we cannot influence it.
The same is true for $F_\mathrm{NUV, min}$, where again higher values increase the evidence as the binomial distribution for $H_\mathrm{1}$ gets increasingly skewed and shifted away from the one for $H_\mathrm{null}$.
However, one might `cherry-pick' exoplanets for which a biosignature test is performed based on \textit{a priori} available contextual information~\citep{catling2018exoplanet} in order to maximize the science yield of investing additional resources.
For instance, the distribution of $F_\mathrm{NUV}$ in the planet sample can be influenced by the survey strategy, and a targeted sampling approach could favor extreme values.% of $F_\mathrm{NUV}$ in the sample selection.
Figure~\ref{fig:semian_selectivity} shows how the probability of obtaining true strong evidence for $H_\mathrm{1}$ scales with selectivity $s$, where $s\in]-1,1[$ such that $F_\mathrm{NUV} \sim Beta(1/10^s,1/10^s)$.
Here, $s=0$ corresponds to a random uniform distribution.
Compared to this case, a high selectivity can increase the probability of obtaining true strong evidence to $\gtrsim \SI{85}{\percent}$ for large samples.

\begin{figure*}
    \script{semian_true_evidence.py}
    \begin{centering}
        \includegraphics[width=\hsize]{figures/semian_selectivity}
        \caption{Scaling of the probability of obtaining true strong evidence with sample selectivity. Left: Sampling distribution for different selectivity parameters $s$. Right: Resulting \mbox{P(true strong evidence)} (random, uniform $f_\mathrm{life}$, $F_\mathrm{NUV, min}$). Sampling more extreme values of $F_\mathrm{NUV}$ is more likely to yield strong evidence.}
        \label{fig:semian_selectivity}
    \end{centering}
\end{figure*}




\subsection{Survey simulations with Bioverse}

In a magnitude- and volume-limited sample of a transit survey, the host star distribution will be skewed toward later spectral types and dominated by M~dwarfs (see Figure~\ref{fig:inhabited_FGKM}).
Due to how the \gls{HZ} scales with spectral type, by far most transiting \glspl{EEC} occur around M~dwarfs.
Their \gls{NUV} fluxes are generally highest at early times $\lesssim\SI{100}{\mega\year}$.
These host stars, in particular late subtypes, also provide extended periods of increased \gls{NUV} emission that overlap with times when some of these planets occupy the \gls{HZ} (see Figure~\ref{fig:hz_nuv_evo}), our requirement for abiogenesis (compare Equation~\ref{eq:hypothesis-uv}).
Because of that, all inhabited planets in our magnitude- and volume-limited sample orbit M~dwarfs.

Here, we are interested in the statistical power of a transit survey with a realistic sample selection and size.
In the following, we fix the sample size to \var{N_nautilus} and consider two different survey strategies targeting FGK and M~dwarfs, respectively.
We further investigate the sensitivity of the survey to the a priori unknown threshold \gls{NUV} flux $F_\mathrm{NUV, min}$ and the abiogenesis rate $f_\mathrm{life}$.

\subsubsection{Selectivity of simulated transit surveys}
In Section~\ref{sec:results-semianalytical}, we demonstrated that the probability of obtaining true strong evidence for the hypothesis that life only originates on planets with a minimum past \gls{NUV} flux is sensitive to the distribution of sampled past \gls{NUV} fluxes, i.e., the selectivity of the survey (compare Figure~\ref{fig:semian_selectivity}).
%The maximum \gls{NUV} distribution of our generic transit survey is rather unimodal with a median around \SI{500}{\erg\per\second\per\centi\meter\squared} (see Figure~\ref{fig:nuv_distribution}), leading to a selectivity of roughly \var{selectivity_transit_volumelim}.
For both surveys targeting M dwarfs and those targeting FGK dwarfs, the maximum \gls{NUV} distribution is rather unimodal.
Applying the approach from Sect.~\ref{sec:results-semianalytical} of fitting a Beta function to the distribution, we find rather low selectivities (see Figure~\ref{fig:nuv_distribution}).
\begin{figure*}
    \script{nuv_distribution.py}
    \begin{centering}
        \includegraphics[width=\hsize]{figures/nuv_distribution}
        \caption{Distribution of maximum past \gls{NUV} flux in transit surveys targeting FGK and M~stars, respectively. The best-fit beta distributions (gray) correspond to selectivities of $s_\mathrm{FGK} = $\var{selectivity_FGK} and $s_\mathrm{M} = $\var{selectivity_M}. Red areas show inhabited planets for a threshold \gls{NUV} flux of $F_\mathrm{NUV, min} = \var{NUV_thresh}\,\SI{}{\erg\per\second\per\centi\meter\squared}$ and an abiogenesis rate of $f_\mathrm{life} = \var{f_life}$.}
        \label{fig:nuv_distribution}
    \end{centering}
\end{figure*}


\subsubsection{Expected biosignature pattern in a transit survey}
A representative recovery of the injected biosignature pattern is shown in Figure~\ref{fig:detections_uv}.
There, we assumed an abiogenesis rate of $f_\mathrm{life} = \var{f_life}$ and a minimum \gls{NUV} flux of $F_\mathrm{NUV, min} = \var{NUV_thresh}\,\SI{}{\erg\per\second\per\centi\meter\squared}$.
All injected biosignatures are assumed to be detected, and the maximum \gls{NUV} flux is estimated from the host star's spectral type and age with an uncertainty corresponding to the intrinsic scatter in the \gls{NUV} fluxes in \citet{Richey-Yowell2023}.
\begin{figure*}
    \script{detections_uv.py}
    \begin{centering}
        \includegraphics[width=\hsize]{figures/detections_uv}
        \caption{Recovered biosignature detections in the \gls{NUV} flux-biosignature occurrence space. The dashed line denotes a generic threshold \gls{NUV} flux $F_\mathrm{NUV, min} = \var{NUV_thresh}\,\SI{}{\erg\per\second\per\centi\meter\squared}$.}
        \label{fig:detections_uv}
    \end{centering}
\end{figure*}
This leads to a distribution of biosignature detections with detections increasingly occurring above a threshold inferred NUV flux.
In this example case, the few biosignature detections in the FGK sample lead to a higher evidence ($dlnZ_\mathrm{FGK} = \var{dlnZ_FGK}$) than in the M dwarf sample ($dlnZ_\mathrm{M} = \var{dlnZ_M}$), where the majority of planets are above the threshold \gls{NUV} flux.

\begin{figure}
    \script{inhabited_aafo_nuv_thresh.py}
    \begin{centering}
        \includegraphics[width=\hsize]{figures/inhabited_aafo_nuv_thresh}
        \caption{Fraction of inhabited planets for different threshold \gls{NUV} fluxes under the UV hypothesis if the abiogenesis rate is \var{f_life}. Shaded regions denote 90 percent confidence intervals of randomized sample generations, and dashed lines correspond to samples including non-transiting planets.
        For all samples, the fraction of inhabited planets drops sharply with increasing threshold \gls{NUV} flux due to the combined effects of never receiving sufficient \gls{NUV} flux for abiogenesis or receiving it before entering the \gls{HZ}.}
        \label{fig:inhabited_aafo_nuv_thresh}
    \end{centering}
\end{figure}
Figure \ref{fig:inhabited_aafo_nuv_thresh} shows the fraction of inhabited planets under the UV hypothesis for different threshold \gls{NUV} fluxes and for a high abiogenesis rate of \var{f_life}.
This fraction decreases sharply with increasing threshold flux, as fewer planets receive sufficient \gls{NUV} flux for abiogenesis.
Another effect responsible for this drop is that some planets receive the required \gls{NUV} flux only before entering the \gls{HZ} -- this is especially likely for M~dwarfs.
For the FGK sample, the fraction of inhabited planets drops at lower threshold fluxes than for the M dwarf sample.


\subsubsection{Statistical power for a transit survey and sensitivity on astrophysical parameters}
We now investigate the sensitivity of the achieved statistical power of our default transit survey to the a priori unconstrained threshold \gls{NUV} flux $F_\mathrm{NUV, min}$ and the abiogenesis rate $f_\mathrm{life}$.
Figure~\ref{fig:powergrid} shows the statistical power as a function of these parameters for a sample size of $N=\var{N_nautilus}$.
Values of $F_\mathrm{NUV, min}$ that lie between the extrema of the inferred maximum \gls{NUV} flux increase the achieved statistical power of the survey, as in this case the dataset under the alternative hypothesis $H_1$ differs more from the null hypothesis.
The same is true for the abiogenesis rate $f_\mathrm{life}$, where higher values increase the evidence for $H_1$.

\begin{figure*}
    \script{powergrid.py}
    \begin{centering}
        \includegraphics[width=\hsize]{figures/powergrid}
        \caption{Statistical power as a function of threshold \gls{NUV} flux and abiogenesis rate. Even for a large sample (here: $N=\var{N_nautilus}$), a high statistical power of the transit survey requires high abiogenesis rates $f_\mathrm{life}$.
         Intermediate values of $F_\mathrm{NUV, min}$ are more likely to yield strong evidence than extreme values; the sensitivity of the M~dwarf sample extends into the low \gls{NUV} flux end.
         }
        \label{fig:powergrid}
    \end{centering}
\end{figure*}



\section{Discussion}
\label{sec:discussion}
%''Which natural processes best explain how living matter spontaneously appears from nonliving matter?''~\citep{Malaterre2022}
%WE consider life ``as we know it''
\subsection{Constraining power for the origins of life as a function of biosignature location}
\todo[inline]{How does the location of biosignature detections impact the credibility of OOL scenarios?}
\todo[inline]{biosignature on M~dwarf planet vs. FGK: Impact on UV flux requirement?}
The cyanosulfidic scenario, in particular its predicted existence of a minimum \gls{NUV} flux required for prebiotic chemistry, offers an opportunity to test an origins of life hypothesis with a statistical transit survey sampling planets with varying \gls{NUV} flux histories.
It comes to no surprise that the success rate of such a test is sensitive to the sample size of the survey and to the occurrence of life on temperate exoplanets.
As we have shown, the statistical power of this test also depends on the distribution of past \gls{NUV} fluxes in the sample and on the required threshold flux.
Optimizing the survey to sample a wide range of NUV flux values, particularly at the extremes, can enhance the likelihood of obtaining strong evidence for or against the hypothesis.
Intermediate values of the threshold \gls{NUV} flux are more likely to yield strong evidence than extreme values, as the dataset under the alternative hypothesis $H_1$ differs more from the null hypothesis in this case while still being sufficiently populated.
The required threshold flux is, of course, a priori unknown and we cannot influence it.
If, however, better theoretical predictions for the required \gls{NUV} flux for abiogenesis become available, the survey strategy can be further optimized, for instance by targeting planets that are estimated to have received a \gls{NUV} flux slightly below and above this threshold.

An interesting  aspect lies in the distribution of host star spectral types:
Under the \gls{NUV} hypothesis, the occurrence of life is expected to correlate with the host star's spectral type, with late-type M~dwarfs being favored due to their higher \gls{NUV} fluxes in the \gls{HZ}~\citep{Richey-Yowell2023}.
However, the distribution of maximum past \gls{NUV} fluxes in this sample is narrow in this sample, which may limit the constraining power of the survey depending on the (unknown) threshold \gls{NUV} flux.
An M~dwarf sample may help to test the high \gls{NUV} flux end of the hypothesis; a higher occurrence of biosignatures here would support the hypothesis that a higher NUV flux is beneficial or necessary for life.

FGK stars, on the other hand, show a wider distribution of maximum past \gls{NUV} fluxes in the \gls{HZ}, which may increase the likelihood of obtaining strong evidence for or against the hypothesis.
With this sample, the survey will be more sensitive to the low \gls{NUV} flux end of the hypothesis.
A lack of biosignatures on these planets would support the \gls{NUV} hypothesis, whereas their presence might suggest that lower NUV fluxes are also sufficient for abiogenesis or indicate different abiogenic pathways. CITE






\subsection{What do we learn from a single biosignature detection?}
\todo[inline]{discuss constraining power on OOL of a convincing biosignature detection on a single planet, depending on the position of the planet in the parameter space we explored.}

\subsection{Sampling strategy for testing a predicted minimum \gls{NUV} flux}
In Sect.~\ref{sec:results}, we show that testing the hypothesis of a minimum past \gls{NUV} flux required for abiogenesis suffers from so-called `nuisance' parameters-- unspecified parameters entering the equation of a statistical test-- that render inference difficult.
Here, these parameters are the unspecified value of the \gls{NUV} threshold hypothesized to exist under $H_1$, and the unknown probability of detection of biosignatures should life have emerged on a planet $f_{\mathrm{life}}$.
While the history of a planet's received UV flux is difficult to infer~\citep[e.g.,][]{Richey-Yowell2023}, the maximum value of \gls{NUV} flux that a planet was exposed to during its existence might be used as a proxy.
Indeed, the distribution of the number of planets with detected biosignature in a particular sample of planets with determined maximum \gls{NUV} values $F_{\mathrm{NUV}}$ depends on both the values of $F_{\mathrm{NUV},min}$ and $f_{\mathrm{life}}$ as shown in equation \ref{eq:semian:likelihoodH1}.
Avoiding the loss in statistical power due to nuisance parameters is a common difficulty in statistical testing, and is the motivation behind elaborate test statistics.
In the exercise that we do here, we project what could be a test performed by a future observer who disposes of a sample of exoplanets with known past maximum \gls{NUV} exposure and for which biosignature detection has been attempted.
This is necessarily reductive as the future observer will have more knowledge about experimental conditions and will therefore be able to use this information to guide hypothesis testing.
For instance, we have made the choice in Sect.~\ref{sec:results} to consider the total number of detected biosignatures as our summary statistic (equation \ref{eq:semian:likelihoodH1), which is not `sufficient' to infer $F_{\mathrm{NUV},min}$ and $f_{life}$ separately.
However, by conditioning the Bayes factor to these variables (equation \ref{eq:bayesfactor}), we calculate the probability distribution of the Bayesian evidence in favor of $H_1$.
In doing so, we may evaluate how evidence depends on the uncertainty over these unknown parameters in general terms, without assuming which particular test a future observer might choose to actually perform over real data when available.
This rather unrefined analysis allows us to provide the intuitive insight that target selection may strongly affect the conclusiveness of one's future test aimed at assessing whether a planet's exposure to \gls{NUV} affects the probability of abiogenesis.
In particular, a future observer might prefer prioritize sampling extreme values of this variable in order to maximize statistical power.
This clashes with observational constraints, as the distribution of planets that we can observe and for which detection of biosignature can be attempted might not be independent from \gls{NUV} flux history as the latter is function of stellar mass.
Hence, for the future observer, selectivity and sample size are in a trade-off conflict.
Therefore, quantifying this trade-off conflict in terms of resulting expected evidence yield is relevant in order to design observational strategies.
Our analysis shows that regardless of selectivity, sample sizes smaller than 50 likely result in inconclusive tests, and that increasing selectivity towards extreme $F_{\mathrm{NUV}}$ offers limited inference gains compared to the uniform case ($s=0$; Fig. \ref{fig:semian_selectivity}).
However, our results show that samples of exoplanets with values of $F_{\mathrm{NUV}}$ even only weakly distributed around intermediate values may prevent inference entirely.
Hence, we argue that selection of a sample of planets with uniformly or extreme-skewed (as opposed to center-skewed) distributed $F_{\mathrm{NUV}}$  should be considered a requirement of any future attempt at testing the hypothesis that \gls{NUV} flux contributes to determining the likelihood of abiogenesis.

We stress that a future observer will likely be able to leverage standard frequentist testing methods which may be more likely to yield conclusive results (whether in favor or in disfavor of $H_1$).
For instance, it would make sense when presented with a sample of planets with various values of $F_{\mathrm{NUV}}$ and for which biosigature detection is attempted to categorize them into two classes: those for which a biosignature is detected, and those for which this is not the case.
Then, the distribution of $F_{\mathrm{NUV}}$ in each of those categories is compared under the null hypothesis that they are the same.
In this case, a non-parametric test could be preferred, such as a two-sample Kolmogorov-Smirnov test or perhaps a Mann-Whitney $U$ test.
% Is this what you end up looking at using BIOVERSE ? If so, then that should flow nicely with the rest.

%\todo[inline]{Martin: is the given ref about inference difficulty only, or does it also support the statement that max NUV can be inferred? In any case, I beleive that there should be a statement here about whether max past NUV can be more easily inferred, how (star mass ?), and how it relates to a scalar variable pertaining to the probability ogf abiogenesis. e.g. if max past NUV is below required threshold, then no abiogenesis under H1, but if it is over, there might remain the uncertainty in the timing of this value of NUV ? Can we know if the timing matches the planet being in the HZ?}

%We show in Sect.~\ref{sec:results} that the constraining power for testing the hypothesis of a minimum past \gls{NUV} flux required for abiogenesis is sensitive on the occurrence of life, the value of this threshold flux, the sample size, and the distribution of sampled past \gls{NUV} fluxes.
%In particular the last parameter offers an opportunity to optimize the survey strategy: Although constraints on a planet's UV history have generally large uncertainties~\citep[e.g.,][]{Richey-Yowell2023}, the likely maximum flux of a planet can be inferred and used as a proxy.

%Sampling more extreme (low and high) values for the maximum flux increases the probability of obtaining true strong evidence for the hypothesis.
%For large sample sizes $\gtrsim 200$, this strategy can push this probability into the \SI{90}{\percent} range.
%\todo[inline]{and what about the probability of getting true strong evidence for H0?}: AA reply to Martin. In BF analyses, the evidences for H0 and H1 are symmetric. We just look at the probability of strong true evidence, whether in favor of H0 or H1. If that is not what the code is doing, it is what it should be doing




\subsection{Contextual support for potential biosignature detections}
\todo[inline]{discuss how planetary context may impact credibility of tentative biosignature detections (e.g., if there is a good fit with a predicted OOL pattern; or the opposite: the planetary context does not fit well to any OOL scenarios)}
As we have shown in Section~\ref{sec:results}, the interplay of \gls{NUV} evolution and \gls{HZ} occupancy strongly favors late spectral types for abiogenesis via the cyanosulfidic scenario.
This strong predicted correlation between stellar spectral type and the occurrence of life can be used to falsify \gls{ool} scenarios.
The constraint of this context is particularly strong if a candidate biosignature is detected on a planet orbiting an earlier type star, i.e., where it is unexpected in the context of this \gls{ool} scenario.



\subsection{Caveats}
\subsubsection{Atmosphere transmission}
Theoretical work suggests that the atmosphere of prebiotic Earth was largely transparent at near-UV wavelengths with the only known source of attenuation being Rayleigh scattering~\citep{Ranjan2017,Ranjan2017c}.
We thus approximated surface UV flux using top-of-atmosphere fluxes.
This represents a conservative approach, since any planet that fails to meet the irradiance criterion receives even lower near-UV radiation at its surface. \todo[inline]{SUKRIT please specify as neccessary}

\subsubsection{Stellar flares}
Our assumptions on past UV flux neglect the contribution of stellar flares, which may be hypothesized as an alternative source of UV light~\citep{Ranjan2017}.
This concerns mainly ultracool dwarfs, due to their low quiescent emission and high pre-main sequence stellar activity~\citep{Buccino2007,West2008}.
Recent work indicates that the majority of stars show inadequate activity levels for a sufficient contribution through flares~\citep{Glazier2020,Ducrot2020,Guenther2020}.
The biosignature surveys we simulated here may test the hypothesis of sufficient UV radiation via stellar flares.

%\subsubsection{Diurnal cycles}
%\todo[inline]{We assume that tidal locking implies that there are no wet-dry cycles. Challenge this assumption.}


\section{Conclusions and future work}
We have investigated the potential of upcoming exoplanet surveys to test the hypothesis -- motivated by the cyanosulfidic origins-of-life scenario -- that a minimum past \gls{NUV} flux is required for abiogenesis.
To this end, we first employed a semi-analytical Bayesian analysis to estimate probabilities of obtaining strong evidence for or against this hypothesis.
We then used the Bioverse framework to assess the diagnostic power of realistic transit surveys, taking into account exoplanet demographics, time-dependency of habitability and \gls{NUV} fluxes, observational biases, and target selection.

Our main findings are:
\begin{enumerate}
    \item Under the \gls{NUV} hypothesis, the fraction of inhabited planets in a transit survey is sensitive to the threshold \gls{NUV} flux and is expected to drop sharply for required fluxes above a few hundred~\SI{}{\erg\per\second\per\centi\meter\squared}.
    \item The \gls{NUV} hypothesis should lead to a correlation between past \gls{NUV} flux and current occurrence of biosignatures that may be observationally testable.
    \item The required sample size for detecting this correlation depends on the abiogenesis rate on temperate exoplanets and the distribution of host star spectral types in the sample; in particular their past maximum \gls{NUV} fluxes.
%        A direct imaging mission with the \gls{HWO} will require characterizing XXX planets ...
    \item If the predicted \gls{NUV} correlation exists, obtaining strong evidence for the hypothesis is likely ($\gtrsim \SI{80}{\percent}$) for sample sizes $\geq 100$ if the abiogenesis rate is high ($\gtrsim \SI{50}{\percent}$) and if no very high \gls{NUV} fluxes are required.
    A survey strategy that targets extreme values of inferred past \gls{NUV} irradiation increases the diagnostic power.
    \item Samples of planets orbiting M~dwarfs overall yield higher chances of successfully testing the \gls{NUV} hypothesis and are more likely to yield biosignature detections under this hypothesis.
%    \item Abiogenesis in the cyanosulfidic scenario prefers late host star spectral types due to the higher \gls{NUV} fluxes in their habitable zones. However, due to a wider distribution of maximum past \gls{NUV} fluxes in the \gls{HZ} of FGK stars, a transit survey targeting FGK systems is more likely to obtain strong evidence for or against the \gls{NUV} hypothesis.

%     in a sample of transiting planets. Any potential biosignature detection on a planet around a star other than an M~dwarf would disvafor this scenario and any other \gls{ool} theory requiring extended periods of high \gls{NUV} irradiation.
\end{enumerate}

Overall, our work demonstrates that upcoming exoplanet surveys have the potential to test the hypothesis that a minimum past \gls{NUV} flux is required for abiogenesis.
More generally, we found that models of the origins of life provide hypotheses that may be testable with near-future exoplanet surveys.
Our work highlights the importance of understanding the context in which a biosignature detection is made, which can not only help to assess the credibility of the detection but also to test competing theories of the origins of life on Earth and beyond.

%\begin{acknowledgments}
\section*{Acknowledgments}
    The authors thank Kevin Heng, Dominik Hintz, Chia-Lung Lin, and Rhys Seeburger for insightful discussions.
%    We are grateful to ...
%    mention any people who gave comments in early-on meetings
    We thank the anonymous referee for providing constructive critical feedback that helped to improve this manuscript.
    This material is based upon work supported by the National Aeronautics and Space Administration under Agreement No. 80NSSC21K0593 for the program ``Alien Earths''.
    The results reported herein benefited from collaborations and/or information exchange within NASA’s Nexus for Exoplanet System Science (NExSS) research coordination network sponsored by NASA’s Science Mission Directorate.
    This work has made use of data from the European Space Agency (ESA) mission \gaia\ (\url{https://www.cosmos.esa.int/gaia}), processed by the \gaia\ Data Processing and Analysis Consortium (DPAC, \url{https://www.cosmos.esa.int/web/gaia/dpac/consortium}). Funding for the DPAC has been provided by national institutions, in particular the institutions participating in the \gaia\ Multilateral Agreement.
    T.L. was supported by the Branco Weiss Foundation, the Netherlands eScience Center (PROTEUS project, NLESC.OEC.2023.017), and the Alfred P. Sloan Foundation (AEThER project, G202114194).
%    ...
\end{acknowledgments}

\section*{Author contributions}
M.S., D.A., and S.R.\ conceived the project, planned its implementation, and interpreted the results.
M.S.\ developed the planetary evolution component to \bioverse, carried out the hypothesis tests and statistical analyses, and wrote the manuscript.
D.A.\ leads the ``Alien Earths'' program through which this project is funded, helped to guide the strategy of the project, and provided text contributions.
A.A.\ carried out the semi-analytical computations regarding the correlation of past UV flux and biosignature occurrence.
S.R.\ advised on planetary \gls{NUV} flux evolution and the cyanosulfidic scenario of the origins of life.
R.F.\ wrote the initial draft of the Introduction and advised on the evolutionary biology aspects of the project.
K.H.-U.\ contributed to the \bioverse\ software development and simulations.
%M.S.\ wrote the manuscript; ... and ... provided text contributions.
T.L.\ supported the selection of testable hypotheses and provided text contributions to the initial draft.
S.M.\ advised on the scope of the project and supported the selection of testable hypotheses.
All authors provided comments and suggestions on the manuscript.


%Author affiliations (tbd):
%Martin Schlecker1, Daniel Apai1,2, Antonin Affholder4, Sukrit Ranjan2, Regis Ferriere4, Kevin Hardegree-Ullman1, Tim Lichtenberg3, and Stephane Mazevet5
%1 Steward Observatory and Department of Astronomy, The University of Arizona, Tucson, AZ 85721, USA
%2 Lunar and Planetary Laboratory, The University of Arizona, Tucson, AZ 85721, USA
%3 Kapteyn Astronomical Institute, University of Groningen, PO Box 800, 9700 AV Groningen, The Netherlands
%4 Department of Ecology and Evolutionary Biology, University of Arizona, Tucson, AZ, USA
%5 Observatoire de la Côte d’Azur, 96 Boulevard de l’Observatoire, F-06300 Nice, France


\section*{Reproducibility}

%! suppress = MissingBibliographystyle
\bibliography{bib,coauthors}

\end{document}
